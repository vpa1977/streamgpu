\chapter{k-Nearest Neighbours}

\section*{Problem Statement}
 
k-Nearest Neighbours method is a non-parametric method used for the classification and regression. It computes a given instance distance to the examples with the known label and either provides a class membership for the classification which is a class most common among nearest neighbours or an object property value which is an average of the nearest neighbours. 
The error rate bound by twice the Bayes error if the number of examples approaches infinity.
The naive approach computes distance to each example and has computational complexity  $ O(N^{d}) $ where N - number of examples and d - cardinality of the example.
The method optimizations deal with organizing the search space to reduce
complexity associated with distance calculation. Examples would be branch
and bounds methods such as kd-tree that partition search space, and approximate
methods, e.g. locality sensivity hash that simplifies the distance function by
mapping instances into lower dimensional space preserving their pair-wise distances within the certain error margin. 

\paragraph*{Exhaustive search} 
The exhaustive search approach consists of distance
calculation and selection phase.The distances to the query a computed as a
vector-matrix multiplication or if several queries are processed at once as a
matrix-matrix multiplication. GPU implementation of those routines is available
as a part of libraries implementing BLAS\cite{cuBLAS}\cite{clBlas}\cite{ViennaCL}. The selection phase finds nearest to the query out of all the computed distances. Sismanis et.
al\cite{Sismanis2012} provide time complexity of reduced sort algorithms and evaluates their performance on GPU, proposes to interleave distance calculation and sorting phases to hide latency - the data for the distance calculation should be offloaded to GPU while it performs the sorting phase. The input data in the brute-force approach is partitioned according to the GPU memory capabilities and does not use examples's spatial information.

\paragraph*{Space parititioning methods} 

The space paritioning techinques are widely used to limit number of distance
calculations needed for nearest neighbour search. The most famous are $ k-d $
tree, ball tree and cover tree.

\subparagraph*{K-D Trees}
 
The $ k-d $ tree \cite{(Friedman et al., 1977} is a balanced binary tree where each node
represents a set of points $ P\in\langle p_1 \cdot p_n \rangle $ and its
children are disjoint and almost equals sized subsets of $ P $. The tree is constructed
top-down, the initial set of points is split along the widest dimension or using
other criteria until the predefined number of points in child nodes is reached.
The tree can be constructed in $ O(n log n)$ time and occupies linear space.
Weber $ et al $\cite{Weber:1998:QAP:645924.671192} have shown that  $ k-d $
tree is outperformed by the exact calculation at moderate dimensionality (
$ n > 10 $ ) and results in full processing of the data points if the number of
dimensions is large enough.
The $ k-d $ tree requires $ N \gg 2^k $ points to be more effective than exhaustive
search.

 \begin{figure}[htp]
 	\begin{lstlisting}
		tree_node create_tree(pointList, level)
		{
			int dim = select_dim(pointList); // select split dimension according to 
											 // pre-defined criteria, e.g. level mod total_dimensions
			splitVal = select_split_value( pointList, dim); // select split value
															// according to pre-defined criteria
															// e.g. median value of point[dim]
			left = {};
			right = {}
			for (point : pointList ) 
			{
			     if (point[dim] > splitVal) 
			        right += point;
			     else
			     	left += point;
			}
			node = {
				.location = splitVal,
				.dim = dim,
				.left = create_tree(left, level + 1),
				.right = create_tree(right, level + 1) 
			};
			return node;
		}
		
		void search(Heap nearest_neighbours, tree_node root, point p) 
		{
			if (root.is_leaf()) 
			{
				nearest_neighbours.update(root);
			}
			else
			{
				split = root.location;
				dim = root.dim;
				if (p[dim] < split ) // search "closest" node
					search(nearest_neighbours, root.left, p) 
				else
					search(nearest_neighbours, root.right, p)
					
				distance_to_split_plane = abs(split-p[dim]); 
				distance_to_point = abs(nearest_neighbours.furtherst_point()[dim]-p[dim]) 
				if (distance_to_point >=distance_to_split_plane) // outer radius of NN heap
				intersects the split plane
				{
					if (p[dim] < split ) 
						search(nearest_neighbours, root.right p) 
					else
						search(nearest_neighbours, root.left, p)
				}
					
				
			}
		}
		
	\end{lstlisting}
	\caption{\label{k-d tree pseudocode} k-d tree construction and NN-search
	pseudocode}
\end{figure}
The listing of the $ k-d $ tree construction and nearest neighbours search
pseudocode is shown in the Figure \ref{k-d tree pseudocode}.
The parallel $ k-d $ tree construction on GPU utilizes breadth-first
approach\cite{Zhou}\cite{Shevtsov_highlyparallel} - the $ k-d $ tree is
constructed top-down with the split criteria computed in parallel for all nodes
at the specific level.
The standard nearest neighbours search using k-d trees does not benefit much
from the GP GPU parallelism due to the branch divergence and irregular memory access
patterns\cite{gieseke2014buffer}. The $ k-d $ tree search approach presented by
Gieske et. al focuses on parallel execution of nearest neighbour queries in a
lazy fashion. The query points are accumulated in the leaf nodes of the kd-tree until enough of them is present and then processed as a batch. This solves an issue of the GPU underutilization and low performance if leaf nodes are processed sequentially for each example\cite{gieseke2014buffer}.

This work proposes parallel evaluation of the $ k-d $ tree for a single query.
The tree is built as normal using algorithm in Figure \ref{k-d tree
pseudocode}.Then each node split is evaluated against data points and a data
point hash code is composed $ C = {c_0, \ldots , c_p} bits, p =
2^{tree_levels-1} - 1 $. $c_i$ is $ 1 $ if split criteria is evaluated to true and false otherwise. 
The query is performed by calculating a point hash code and distances to each
split plane. 
The algorithm then calculates distances to the points with the matching hash
code and if the distance to the furtherst $k-th$ point is greater than distances
to the split planes, searches points in the opposite splits by flipping
corresponding bits and repeating the brute-force distance calculation.
The distance calculation is performed by assigning a workgroup to calculate the
single distance thus the thread divergence is not an issue. 

\subparagraph*{Random Projection Trees}

The $ k-d $ tree provides effective partitioning mechanism for
low data dimensionality. Many machine learning problems that are expressed
in high dimensional space has lower intrinsic dimension as shown in Figure
\ref{IntrinsicDimension}. 
\begin{figure}[htp] \resizebox{1\textwidth}{!}{
		\includegraphics[width=\textwidth]{data_with_low_intrnsic_dimension.png}
		}
	\caption{\label{IntrinsicDimension}Distributions with low intrinsic dimension. The purple areas in these figures indicate regions in which
the density of the data is significant, while the complementary white areas indicate areas where data density is very
low. The left figure depicts data concentrated near a one-dimensional manifold. The ellipses represent mean+PCA
approximations to subsets of the data. Our goal is to partition data into small diameter regions so that the data in each
region is well-approximated by its mean+PCA. The right figure depicts a situation where the dimension of the data is
variable. Some of the data lies close to a one-dimensional manifold, some of the data spans two dimensions, and some
of the data (represented by the red dot) is concentrated around a single point
(a zero-dimensional manifold). Reproduced from Learning the structure of
manifolds using random projections by Freund Yoav et
al.\cite{FreundManifoldRandomProjection} }
\end{figure}
Random projection tree exploits this fact by splitting data along randomly
chosen unit vectors as opposed to splitting along dimension
axises in $ k-d$ tree method as shown in
Figure \ref{RPTree}\cite{FreundManifoldRandomProjection}. The method performs a
one dimensional random projection of the data points and splits them at 
the median of the projections.  

\begin{figure}[htp] \resizebox{1\textwidth}{!}{
		\includegraphics[width=\textwidth]{kd_tree_rp_tree.png}
		}
	\caption{\label{RPTree} Left: Partitioning produced by k-d tree. Right:
	Partitioning produced by Random Projection Tree. Reproduced from Learning the
	structure of manifolds using random projections by Freund Yoav et al.\cite{FreundManifoldRandomProjection} }
\end{figure}

 \begin{figure}[htp]
 	\begin{lstlisting}
 		tree_node random_tree_max(pointList, num_dimensions) 
 		{
 			v = random_vector(num_dimensions); 
 			
 			x = pointList[ random() ];
 			y = max (distance( y in pointList, x) );
 			sigma = uniform_random(-1;1) * 6 * distance(x,y) / sqrt(num_dimensions);
 			split = median ( dot(v, x in pointList)+ sigma ) ;
 			left = {}
 			right = {}
 			for (x in pointList) 
 			{
 				if (dot(v,x) <= split) 
 					left += x;
 				else
 					right +=x;
 			}
			node = {
				.vector = v,
				.split = split,
				.left = create_tree(left, num_dimensions),
				.right = create_tree(right, num_dimensions) 
			};
			return node;
 		}
 		
 		tree_node random_tree_mid(pointList, num_dimensions, c) 
 		{
 			diameter = max( distance(x in pointList, y in pointList)); 
 			avg_diameter = mean(distance(x in pointList, y in pointList));
 			if ( diameter <= c* avg_diameter) 
 			{
 				v = random_vector(num_dimensions);
 				split = median( dot(x in pointList, v) );
 				left = {}
 				right = {}
 				for (x in pointList)
 				{
	 				if (dot(v,x) <= split) 
	 					left += x;
	 				else
	 					right +=x;
	 			}
				node = {
					.rule_type = dotproduct
					.vector = v,
					.split = split,
					.left = create_tree(left, num_dimensions),
					.right = create_tree(right, num_dimensions) 
				};
				return node;
 			}
 			else
 			{
 				meanPoint = mean(x in pointList)
 				split = median( distance(x in pointList, meanPoint);
 				left = {}
 				right = {}
 				for (x in pointList)
 				{
	 				if (distance(x, meanPoint) <= split) 
	 					left += x;
	 				else
	 					right +=x;
	 			}
				node = {
					.rule_type = distance
					.mean = meanPoint,
					.split = split,
					.left = create_tree(left, num_dimensions),
					.right = create_tree(right, num_dimensions) 
				};
				return node;
 				
 			}
 		}
 	\end{lstlisting}
	\caption{\label{random projection tree pseudocode} Random Projection Tree
	Pseudocode\cite{Dasgupta:2008:RPT:1374376.1374452} }
\end{figure}
 	
The random projection tree split rules are presented in Figure \ref{random
projection tree pseudocode}. 

The NN-search procedure is identical to $ k-d $ tree. 
Random Projection Tree construction is computationally more intensive than $
k-d $ tree split, though it consists of GP GPU-friendly operations.
\textit{Explain.}Random Projection Tree NN-search implementation on GP GPU will
have same restrictions as $ k-d $ tree.


\subparagraph*{Ball tree}
Ball tree\cite{Omohundro,1989} is a simplest and oldest data structure suitable
for data represented in arbitrary metric space - data points $ \mathbb{X}
$ with defined distance function $ d : \mathbb{X} \times \mathbb{X} ->
\mathbb{R}^+ $. The data points are bounded by
the tree hierarchy of hyperspheres as opposed to hyper-rectangles in $ k-d $
tree.The hyperspheres are allowed to overlap, the data points are
assigned only to one hypersphere.
At the time of writing there were no widely known GPU-based ball tree
implementation though $k-d$ tree limitations such as branch divergence and
irregular memory access for the tree traversal should apply to the ball tree
algorithm as well.

\subparagraph*{Cover tree}
Cover tree\cite{beygelzimer2006cover} is a $N-$ary tree also suitable for the
data represented in arbitrary metric space. The tree is defined recursively -
at the top level the initial point is arbitrary picked from the data set and
covered by the ball with radius $ 2^i $. The $ i $ is picked so that the
ball covers all data points. On the next level the ball radius is $ 2^(i-1) $ and any points outside this radius generate their own cover
balls. The initial ball is assigned as their parent. The process continues until
each ball contains exactly one point. The cover tree satisfies following
invariants:
\begin{itemize}
  \item Nesting - once point is associated with node  at level $ i $ any
  level $ j < i $ will have  a node associated with this point
  \item Covering - the parent node at level $ i $ covers all its child nodes
  \item Separation - the distance between centers of two distinct nodes at the
  same level is more than node radius.
\end{itemize} 
  
 The nearest neighbours query process is iterative. For the query
point $ p $ and $ Q = { Children(q) : q \in Q_i } $ children of the node $ Q_i $ we form a ball
that includes its children satisfying following condition 
$ Q_{i-1} =  \{q \in Q : d(p,q) \leq min_{q\in Q} d(p,Q) +2^i  \} $
\cite{beygelzimer2006cover}.Thus on each step next level query
radius to the distance to the center of the nearest ball plus its radius $ 2^i $. It should be noted that in practice
the radius is redefined as $ 1.3 ^ i $ as it yields better
results\cite{izbickifaster}.  
The cover tree construction can be performed using breadth-first search approach
to perform contstruction iteratively similiar to $ k-d $ tree iterative construction\cite{sharma2010design}.

\subparagraph*{Nearest Ancestor Tree}
The \textit{nearest ancestor tree} is a simplification of
the cover tree\cite{izbickifaster}.
The \textit{simplified cover tree} is a cover tree without nesting invariant.
Simplified cover tree explicitly defines \textit{level} invariant - each node
has associated integer level and for any child node $ q $ with parent $ p $ $ level(q) = level(p) - 1 $. 
The \textit{nearest ancestor tree} is defined as a simplified cover tree with
nearest ancestor invariant - the maximum distance from the parent to the child
nodes is minimized -  for parent $ q_1 $ its sibling $ q_2 $ and a child node $
p $, $ d(q_1, p) \leq d(q_2, p)$\cite{izbickifaster}.
The nearest ancestor tree requires rebalancing when a new node is inserted if
this invariant is violated. 
The nearest ancestor tree construction requires significantly more distance
calcuation than cover tree, but this is offset by better NN query in most
cases\cite{izbickifaster}.
The parallelisation approach for nearest ancestor tree construction - divide the
data set, construct individual trees and merge the results.  

\subparagraph*{Random Ball Cover}
The hypersphere bounding is used in \textit{Random Ball Cover}
method\cite{Cayton6267877}. It provides a single level metric space cover. For
the initial set of points $ \mathbb{X} = \{ x_1, \ldots , x_n \} $ 
random $ O(\sqrt{n}) $ points are selected to act as representatives and 
$ L $ points to them are attached to them and denoted $ L_r $ -
\textit{ownership list} of representative $ r $.
The method defines two algorithms:
\begin{itemize}
  \item one-shot - representatives are chosen at random, and $ s $ closest
  points are put into ownership list $ L_r $ and the data point may be owned by
  several representatives. To perform a query a distance is computed to each
  representative point, closest one taken and $ k $ closest points are selected from its ownership list. 
  \item exact - the exact search computes distances to all representative points
  and selects closest one $ r_q $. The algorithm then considers representative
  points that satisfy following inequalities:
  \[
  \left\{
  \begin{array}{ll} 
  \ d(r,q)  \leq< 3 d(r_q, q) \\
  \ d(r,q)  < d(r_q, q) + \psi_{r}   
 \end{array}
 \right.
  \]
  where $ d(r,q) $  - distance between representative and query point, $
  \psi_{r} $ - distance between represetnative points and furtherst point from
  ownership list. The ownership lists of those points are brute-force searched
  for $ k $ nearest neighbours.
  
\end{itemize} 

The random ball cover method queries are easily paralellised as they consists
of existing brute-force kNN search and scan stage to establish representative
points of interest for the exact method\cite{Cayton6267877}. 
\textit{TODO - theoretical guarantees}

\paragraph*{Approximate methods}

The nearest neighbours search methods in high dimensional space provides little
benefit over exhaustive search where an exact distance is computed to each point
in the
database\cite{Weber:1998:QAP:645924.671192}\cite{Andoni:2008:NHA:1327452.1327494}.
The approximate methods provide means to overcome this limitation by solving the
problem of finding neighbours whose distance from the query point are at most $
c > 1 $ times greater than distance to the closest neighbour. The approximate
solution can be used to find exact one by computing distance to each approximate
nearest neighbour and choosing closest ones.

\subparagraph*{Random Projection}

Seminal paper by Johnson and Lindenstrauss\cite{johnson84extensionslipschitz}
established that for euclidian spaces any $ x \in \mathbb{R}^n $ can be embedded
into $ \mathbb{R}^k $ with $ k = O(log n/\epsilon^2)$ by projecting $ x $
in $ \mathbb{R}^k $ using projection $ k \times n $ matrix $ \Phi $  without
distorting inter-point distances by more than $ (1\pm\epsilon) $ and $ k \geq
O(log n)$.  Johnson and Lindenstrauss\cite{johnson84extensionslipschitz} has
shown that Johnson-Lindenstrauss condition holds for matrices with following
properties:
\begin{blockquote}
  Spherical symmetry - For any orthogonal matrix $ A \in O(d), \phi A and \phi $ have the same distribution.
  Orthogonality - rows are orthogonal to each other
  Normality - the rows are unit-length vectors
\end{blockquote}\cite{ailon2009fast}

The lower bound of $ k $ was refined by in
several papers\cite{frankl1988johnson}\cite{Dasgupta:2003:EPT:639790.639795}\cite{Indyk:1998:ANN:276698.276876}\cite{Achlioptas:2003:DRP:861182.861189} and with Dasgupta and Gupta\cite{Dasgupta:2003:EPT:639790.639795} proving it to be $ k \geq 4 (\epsilon^2/2 - \epsilon^3/3)^-1 ln n for \epsilon \in (0,1) $.
For high $ n $ this bound will still be too large to effectively employ low
dimensionality search methods such as $ k-d $ tree. An alternative would be to
utilize very low dimensional space and then use disjunction to find desired
result. This approach is essentially an iterative random projection tree search
where the dataset is split along leaf nodes.

The efficient implementation of the random projection-based
algorithms requires a simple approach to construct $ \phi $ and a way to compute
projection faster than naive multiplication of data point by $k \times n $
matrix.
Achlioptas\cite{Achlioptas:2003:DRP:861182.861189} achieved relatively sparse
transformation matrix for random projection by proving that
Johnson-Lindenstrauss condition holds if elements of the projection matrix
are chosen independently according to the following distribution:
\[
\left \{ 
\begin{array}{lll} 
  \ +(n/3)^{-1/2}, P =  1/6 \\
  \ 0, P = 2/3 \\
  \ -(n/3)^{-1/2},P = 1/6   
 \end{array}
 \right.
\]
This method provides a 3-fold speedup over originally
proposed\cite{johnson84extensionslipschitz} since $ 2/3 $ of the transformation
matrix elements are zero.
Nir Ailon and Edo Liberty\cite{ailon2013almost} have developed an almost optimal
random projection transformation with runtime of $ O(n log n)$ as opposed to $ O(kn) $
of the naive implementation. The main idea of the method is the application of
the Heisenberg principle in its signal processing interpretation that both
signal and its spectrum can not be both sharply localized. \textit{TODO:finish
description of 2013 method - multiply by +-1 diagonal matrix, apply FFT and
normalization constant and then randomly select k results to obtain projection.
Reference implementation: https://github.com/gabobert/fast-jlt/tree/master/fjlt}

This implementation is well suited for GPU implementation as it consists of FFT
followed by element-wise operation and scatter parallel primitives.
Figure \ref{Fast Johnson-Lindenstrauss transform} shows comparative performance
of dense matrix multiplication for random projection and Fast
Johnson-Lindenstrauss transform. For the selected hardware configuration the
latter starts to outperform matrix multiplication starting from $ N \geq 16384
$. It should be noted that FLJT has lower memory requirements than
\textit{O(kd))} as it does not require to store dense transformation matrix and
thus capable of projecting higher dimensional data on the same hardware.

 \begin{figure}[h]
 \centering
 \begin{gnuplot} 
  set terminal epslatex color size 15cm,12cm
  set style line 1 lt 1 lw 4 lc rgb '#4682b4' pt -1 
  set style line 2 lt 1 lw 4 lc rgb '#ee0000' pt -1 
  set style line 3 lt 1 lw 4 lc rgb '#008800' pt -1
  set style line 4 lt 1 lw 4 lc rgb '#00aaaa' pt -1
  set style line 5 lt 1 lw 4 lc rgb '#cc0000' pt -1
  set xlabel '$ 2^{Vector Size} $'
  set ylabel '$ Execution Time (msec) $'
  set xtics 1
  set key right top
  plot 'data/random_projection/fast_random_projection_vs_naive.data' using 1:3 with lines ls 1 ti '$ Fast JLT $',\
       'data/random_projection/fast_random_projection_vs_naive.data' using 1:4 with lines ls 2 ti '$ Matrix Multiplication $' 
\end{gnuplot}

 \caption{\label{Fast Johnson-Lindenstrauss transform} Fast
 Johnson-Lindenstrauss transform vs. dense matrix implementation using ViennaCL.
 Test configuration GPU R9 390, CPU AMD A8-7600, AMD Catalyst version 15.20.}
\end{figure}

 

\subparagraph*{Locality Sensivity Hashing}
Locality Sensivity Hashing\cite{Indyk:1998:ANN:276698.276876} is a method that
captalizes on the idea that exist such hash functions $ h(x), x \in
\mathbb{R}^d $ that for  points $p,q \in \mathbb{R}^d $, radius $ R $ and
approximation constant $ c $
\[
\left \{ 
\begin{array}{ll} 
  \|p-q\| \leq R,  P[h(p) = h(q)] \geq P_1  \\
  \|p-q\| \geq cR, P[h(p) = h(q)] \leq P_2   
 \end{array}
 \right.
\]
where probabilty $ P_1 > P_2 $.
The LSH algorithm uses a concatenation of $ M \ll d $ such functions to increase
difference between $P_1$ and $P_2$\cite{Indyk:1998:ANN:276698.276876}. Initially
it was proposed to use Hamming distance as this function satisfies required
properties\cite{Indyk:1998:ANN:276698.276876}. Later it was shown that other
families of hash functions such as $ l_p $
distance\cite{Datar:2004:LHS:997817.997857}, Jaccard coefficient
\cite{Broder:1997:SCW:283554.283370}\cite{Broder:1997:SCW:283202.283370},
angular distance(random projection)\cite{Charikar:2002:SET:509907.509965} are
locally sensitive.
The algorithms selects $ L $ contactenations of the hash functions and uses them
to  transform input dataset points $ v \in \mathbb{R}^d $ into lattice space $
\mathbb{Z}^M$ storing them as $ L $ hash tables. 
The exact query is performed by contactenating contents of the $ L $ bins
corresponding to the hash codes of the query and computing exact distance. The
approximation is obtaining by stopping as soon as $ k $ points in $ cR $
distance from the query point is found. 


\subparagraph*{Space filling curves}
\textit{TODO}
The local neighbourhood of the data point in N dimensional space can be
established via computation of the z-order. The z-order is a space-filling curve computed via sorting points according to their Morton Code. The Morton Code is computed as a bit interleave of the data point coordinates.

The computation of the morton code in original high dimensional space is both impractical due to the curse of dimensionality and  computational complexity - the cost of the code computation and comparison is linear.

Thus it is possible to create a faster approximate k-NN algorithm by exploiting both logarithmic complexity of the fast Johnson-Lindenstrauss transform (logarithmic), reduced number of points needed for distance calculation and overall faster computation in lower dimensional space.  

\textit{TODO:The algorithm below is very similiar to the one described in 
Bi-level Locality Sensitive Hashing for K-Nearest Neighbor Computation/Fast
GPU-based Locality Sensitive Hashing for K-Nearest Neighbor Computation
(gamma.cs.unc.edu/KNN). I have not found use of the z-order curve in the code,
neither the generic morton code algorithm. The improvements could be - use of FJLT as opposed to matrix-vector
multiplication for computing 1-D projections, project to k = sqrt(n) and compute
z-order curve over all dataset as opposed to building e8 lattice in the
individual leaves (not found in code either), use z-order curve to locate
closest tree leaves. }

\begin{itemize}
\item random projection tree partitions the search space. use the parallel tree
search.
\item The NN candidates of the query point can be determined by computing a hash function that preserves relative distance between points
\item Morton code is a hash function computed by interleaving bits of the feature vector
\item Morton code creates a space filling curve (z-order curve) with following property - it never doubles up (definition)
\item Pick k nearest neighbour candidates by following z-order curve, compute bounds of a hypercube and this hypercube will contain all possible NN candidates.
Limit those to ones with distance to the split plane less than 
\end{itemize} 

\textit{TODO:Update text below}

\section*{Algorithm Implementations}
\paragraph*{Brute-Force Approach}
The algorithm maintains a sliding window of examples, calculates distance to the query point for each example and sorts them according to the least distance selecting nearest $k$ neighbours.
\subparagraph*{Sliding Window}
The sliding window is implemented as a FIFO cyclic buffer.The OpenCL
implementation uses partial mapping of the buffer to reduce memory transfers.
\subparagraph*{Distance Calculation}
The distance calculation between query vector and sliding window is a vector by matrix multiplication operation. 
For the dense matrices the implementation performs a serial computation of
the distance, each thread working on its own example. Since all the threads
process attributes in exactly same order there is no wavefront divergence. This
is not the case for the sparse matrices and this approach will cause GPU
underutilization.

The optimal implementation depends on the size of the window and number of
attributes present\cite{Sorensen2011}. For the small instance size ($ \leq 100
$) and windows less than $ 10^{4} $ elements naive implementation will provide the best
solution. Best all around distance calculation should apply different strategies
depending on the windowsize and number of attributes\cite{Sorensen2011}. The
alternatives are presented in the Figure \ref{MatrixVectorSorensen}.

\begin{figure}[htp] \resizebox{1\textwidth}{!}{
		\includegraphics[width=\textwidth]{sorensen_2011.png}
		}
	\caption{\label{MatrixVectorSorensen} Left; Four matrix-vector multiplication
	kernels designed to perform well at different shapes m × n of A. Middle; Tuning
	mesh. Right; Best kernel in practice. The dashed line indicates the minimum
	21504 rows needed in A for full occupancy of the Nvidia Tesla C2050 card in a
	one-thread-per-row kernel. Note the logarithmic axes. Reproduced from
	High-Performance Matrix-Vector Multiplication on the GPU by Hans Henrik
	Brandenborg S{\o}rensen\cite{Sorensen2011}}
\end{figure}


\subparagraph*{Selection}

Alabi, et.al evaluated different selection strategies based on bucket sort algorithm and Merril-Grimshaw implementation of radix sort\cite{Alabi2012}.  

The selection phase may be interleaved with the distance calculation to utilize
both CPU and GPU cores and benefit from the better sort performance on low
window sizes\cite{IntelSortTechRep}.

The work needs to provide several alternative selection
strategies such as Merill-Grimshaw radix sort\cite{journals/ppl/MerrillG11} or
k-bucket Selection\cite{Alabi2012} to provide alternative GPU selection strategy.

Figure \ref{SortSelectPerformance} shows measured performance of different
selection strategies - merge sort from AMD Bolt library\cite{Bolt}, bitonic sort
similiar to reference AMD implementation and radix select based on Alabi, et.
al. implementation\cite{Alabi2012}. 
The CPU sort and choose is a clear winner for small (<65535) window sizes. The
merge sort should be applied to sub $ 2^25 $ windows
 and radix select (with data copy) should be used for larger window sizes. 
\textit{TODO:} The work should investigate in-place radix select and device
enqueue for radix select optimization.
\textit{TODO:} Radix select shows a semi-flat line up to $ 2^23 $ window size. The implementation should be checked for excessive setup.

\begin{figure}[h]
 \centering
 \begin{gnuplot} 
  set terminal epslatex color size 12cm,15cm
  set style line 1 lt 1 lw 4 lc rgb '#4682b4' pt -1 
  set style line 2 lt 1 lw 4 lc rgb '#ee0000' pt -1 
  set style line 3 lt 1 lw 4 lc rgb '#008800' pt -1
  set style line 4 lt 1 lw 4 lc rgb '#00aaaa' pt -1
  set style line 5 lt 1 lw 4 lc rgb '#cc0000' pt -1
  set xrange [10:27]
  set xtics 1
  set xlabel '$ 2^{Vector Size} $'
  set ylabel '$ Execution Time (msec) $'
  set key right top
  set logscale y   
  plot 'data/benchmark_sort/double_sort.data' using 1:3 with lines ls 1 title '$ std::sort $',\
  'data/benchmark_sort/double_sort.data' using 1:4 with lines ls 2 ti '$ Bitonic Sort $',\
  'data/benchmark_sort/double_sort.data' using 1:5 with lines ls 3 ti '$ Merge Sort $',\
  'data/benchmark_sort/double_sort.data' using  1:6 with lines ls 4 ti '$ Radix Select $'
 \end{gnuplot}
 \caption{\label{SortSelectPerformance} Selection Algorithm Performance for
 K=128. Test configuration GPU R9 390, CPU AMD A8-7600, AMD Catalyst version
 15.20.}
\end{figure}



\paragraph*{KD-Tree based k-Nearest Neighbours Search}
The KD-Tree nearest neighbours search is composed of parallel tree construction
over fixed set of instances and evaluation. 
\subparagraph*{Parallel Tree Construction}
The input to the algorithm is the matrix containing example instances and the
desired tree depth.
The tree nodes contain row numbers of the associated instances, split attribute,
split value and  ranges for the associated instances.
The tree split is performed iteratively until the desired tree depth is reached. 
First the ranges are updated for each tree node using kernel based on
Bolt max element kernel\cite{Bolt}. The kernel finds ranges for a
given attribute for each of the nodes in a tree.
A split dimension and point is chosen using the CPU routine.
A mark kernel composes two flag vectors - one for the left child, other for the
right one, a scan is performed to compute offsets and then the child nodes are
populated with the indices of the instances belonging to them.
\textit{TODO Picture}

\subparagraph*{Evaluation}
The evaluation uses same distance calculation kernel as the Naive
Implementation. The evaluation can be either performed sequentially - a
recursive algorithm similiar to the\cite{MOA} implementation or several query
instances can be scheduled at once. 
In this case the progress of each individual query instance is tracked by the
bit vector containing the tree path processed so far, current
node number and current found nearest neighbours.
The algorithm alternates between invoking kernels for leaf node distance
calculation and split plane distance calculation until all query instances are
fully processed.
 \textit{TODO Diagram}

\paragraph*{Fast Johnson-Lindenstrauss transform implementation}
  
\paragraph*{Approximate k-Nearest Neighbours Search}
\textit{TODO}


